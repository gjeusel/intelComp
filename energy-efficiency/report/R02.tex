%----------------------------------------------------------------------------------------
%    PACKAGES AND OTHER DOCUMENT CONFIGURATIONS
%----------------------------------------------------------------------------------------

\documentclass[paper=a4, fontsize=11pt]{article} % A4 paper and 11pt font size

% Brazilian encoding :
\usepackage[brazilian]{babel} % English language/hyphenation
\usepackage[T1]{fontenc} % Use 8-bit encoding that has 256 glyphs
\usepackage[utf8]{inputenc}

\usepackage{graphicx} % to insert png
\usepackage{fourier} % Use the Adobe Utopia font for the document - comment this line to return to the LaTeX default
\usepackage{amsmath,amsfonts,amsthm} % Math packages

\usepackage{lipsum} % Used for inserting dummy 'Lorem ipsum' text into the template

%\usepackage{geometry}
%\geometry{
% %a4paper,
% %total={170mm,257mm},
% %left=25mm,
% %top=25mm,
%}

\usepackage{adjustbox} % for better figure positionning
\usepackage[a4paper]{geometry} % for geometry changes in only one page (\newgeometry{...})
\usepackage[space]{grffile} % allow space in figure path

\usepackage{sectsty} % Allows customizing section commands
%\allsectionsfont{\centering \normalfont\scshape} % Make all sections centered, the default font and small caps

\usepackage{fancyhdr} % Custom headers and footers
\pagestyle{fancyplain} % Makes all pages in the document conform to the custom headers and footers
\fancyhead{} % No page header - if you want one, create it in the same way as the footers below
\fancyfoot[L]{} % Empty left footer
\fancyfoot[C]{} % Empty center footer
\fancyfoot[R]{\thepage} % Page numbering for right footer
\renewcommand{\headrulewidth}{0pt} % Remove header underlines
\renewcommand{\footrulewidth}{0pt} % Remove footer underlines
\setlength{\headheight}{13.6pt} % Customize the height of the header

\numberwithin{equation}{section} % Number equations within sections (i.e. 1.1, 1.2, 2.1, 2.2 instead of 1, 2, 3, 4)
\numberwithin{figure}{section} % Number figures within sections (i.e. 1.1, 1.2, 2.1, 2.2 instead of 1, 2, 3, 4)
\numberwithin{table}{section} % Number tables within sections (i.e. 1.1, 1.2, 2.1, 2.2 instead of 1, 2, 3, 4)

\usepackage{float}% If comment this, figure can moves to next page

\usepackage{indentfirst}% for indenting also the first paragraph after begin section
%\setlength\parindent{0pt} % Removes all indentation from paragraphs - comment this line for an assignment with lots of text

%----------------------------------------------------------------------------------------
%    TITLE SECTION
%----------------------------------------------------------------------------------------

\newcommand{\horrule}[1]{\rule{\linewidth}{#1}} % Create horizontal rule command with 1 argument of height

\title{
\normalfont \normalsize
\textsc{Universidade Federal do Rio de Janeiro} \\ [25pt] % Your university, school and/or department name(s)
\horrule{0.5pt} \\[0.4cm] % Thin top horizontal rule
\huge Aprendizado de Máquina\\Eficiência energética dos edifícios \\ % The assignment title
\horrule{2pt} \\[0.5cm] % Thick bottom horizontal rule
}

\author{Aluno: Guillaume Jeusel} % Your name

%\Professor: Alexandre G. Evsukoff

\date{\normalsize\today} % Today's date or a custom date

\date{\parbox{\linewidth}\bigskip%
  Professor: Alexandre G. Evsukoff \endgraf\bigskip
  Disciplina: Inteligência Computacional \endgraf\bigskip
  {\centering\normalsize\today\endgraf}
  }

\begin{document}

\maketitle % Print the title

\newpage
\tableofcontents % Print table of content

%----------------------------------------------------------------------------------------
%    Introduction :
%----------------------------------------------------------------------------------------
\newpage
\section{Introdução}

\subsection{Problema}
Com uma demanda de energia sempre crescente nosso mundo, o problema de economia de energia é colocado no centro das preocupações.
O conceito de \emph{négaWatt} \cite{ref_negawatt} traduze uma economia de energia devido a uma mudança de comportamento ou de tecnologia usada, e veja essa economia como um ganho.
Além disso, um campo cujo desperdício de energia fica ainda extremamente importante é o edifício.

Por conseguinte, as investigações na área do desempenho energético dos edifícios cresceu muito recentemente;
uma ação prioritária que as sociedades deve ter em mente é a redução do consumo de energia dos novos edifícios, também como a renovação dos antigos.
A propósito, a legislação sobre o desempenho energético dos edifícios é sempre mas exigente,
especificamente nos países europeus com a directiva 2002/91/CE limitando o consumo de energia dos edifícios \cite{ref_europ_energ}.

%------------------------------------------------

\subsection{Conhecimento Prévio}
Para o design desses edifícios, é necessário a computação dos termos chamados \emph{“Heat Load” e “Cooling Load”}
(que pode ser traduzido pelo “carga de aquecimento” e “carga de arrefecimento” respetivamente).
Eles são diretamente ligados à especificação dos equipamentos responsáveis para manter uma temperatura confortável, e então ao consumo energético.
Esses coeficientes são dependentes das características geométricas dos edifícios, como também do clima local e do uso deles (industrial, casal …).

Existem muitos diferentes software de simulação que são eficientes para prever o consumo energético dos edifícios em projeto com uma precisão aceitável.
Eles resolvam as equações diferencias da termodinmica aplicada a uma geometria particular.
No enquanto, essas simulações podem demorar muito tempo, sem mencionar que quando um parmetro é mudado, a simulação deve ser reinicializada desde ao início.

Desse fato, um interesso crescente sobre o uso das técnicas de aprendizado de máquinas nasci.
A ideá é a seguinte: suponho que você tem um banco de dados recente com as características e cargas de um grande número de edifícios,
o uso de estatísticas e aprendizado de máquinas pode reduzir o tempo de computação e facilitar o experimento de diversos parmetros.
Nos podemos pensar até criar um banco de dados com os diferentes resultados de simulação,
e depois prever o desempenho energético de um novo edifício com interpolação dos resultados que nos já temos.

Isto foi a ideá do engenheiro civil \emph{Angeliki Xifara} e do matemático \emph{Athanasios Tsanas} da universidade de Oxford.
Usando o software Ecotect, um conjunto de dado foi criado da simulação do desempenho energético para 768 geometrias de edifícios,
assumindo uma localização em Atena, Grécia e um uso residencial com sete pessoas.
Nos vamos estudar esse banco de dados.

Para ter mais informações sobre as hipóteses de simulação, deve-se referir ao papel deles \cite{ref_Athanasios}.


%----------------------------------------------------------------------------------------
%    Descrição dos dados :
%----------------------------------------------------------------------------------------
\newpage
\section{Descrição dos dados - lembrete}

\subsection{Dados}
O dataset é tirado do web-site UCI – Machine Learning Repository \cite{ref_UCI}.
A figura~\ref{dados_resumo} contem um resumo geral desse conjunto de dados.

\begin{figure}[H] % image flottante, i.e latex selectione le meilleur placement
\begin{center}
\includegraphics[width=\textwidth]{/home/gjeusel/projects/intelComp/energy-efficiency/images/resumo_dos_dados.png}
\end{center}
\caption{Características dos dados}
\label{dados_resumo}
\end{figure}

Para facilitar o estudo das regressões, nos vamos somar a carga de aquecimento e a carga de arrefecimento para ter uma única saída.

\medskip
Ele é composto de 768 registros e tem 8 variáveis de entrada e 1 de saída que são as seguintes:

\begin{table}[H]
\caption{Mathematical representation of the input and output variables}
\medskip
\renewcommand\arraystretch{1.3}
\renewcommand\tabcolsep{0pt}
  \begin{tabular*}{1\linewidth}{@{\extracolsep{\fill}}cccc}
    \hline
    Mathematical representation & Input or output variable & Number of possible values & Unit \\
    \hline
    X1 & Relative Compactness & 12 & None \\
    X2 & Surface Area & 12 & m²\\
    X3 & Wall Area & 7 & m²\\
    X4 & Roof Area & 4 & m²\\
    X5 & Overall Height & 2 & m\\
    X6 & Orientation & 4 & Unknown\\
    X7 & Glazing Area & 4 & m²\\
    X8 & Glazing Area Distribution & 6 & None \\
    y & Heating Load + Cooling Load & 636 & Unknown \\
  \end{tabular*}\par\medskip
\label{tab:priors}
\end{table}

É importante de notar que as variáveis de entradas são descontinuidades.\newline
Um estudo anterior foi realizada, concluindo que o conjunto de dados:
\begin{itemize}
\item não tinha valores ausentes
\item não tinha valores aberrantes (outliers)
\end{itemize}

%------------------------------------------------

\subsection{Distribuições - Histogramas das variáveis não padronizadas}

  \begin{figure}[H] % image flottante, i.e latex selectione le meilleur placement
  \begin{center}
  \includegraphics[width=\textwidth]{/home/gjeusel/projects/intelComp/energy-efficiency/images/Xvar_histograms_part1.png}
  \includegraphics[width=\textwidth]{/home/gjeusel/projects/intelComp/energy-efficiency/images/Xvar_histograms_part2.png}
  \end{center}
  \caption{Histogramas das variáveis de entradas}
  \label{hist_X}
  \end{figure}

Nos podemos comentar que as variáveis “X3 Wall Area”, “X4 Roof Area”, “X7 Glazing Area” e “X8 Glazing Area Distr.” não são bem centradas.
Seja bem de processar com a metodologia de validação cruzada para ser robusto à escolha das partições de treinamento e validação.

  \begin{figure}[H] % image flottante, i.e latex selectione le meilleur placement
  \begin{center}
  \includegraphics[width=0.8\textwidth]{/home/gjeusel/projects/intelComp/energy-efficiency/images/Yvar_summed_histograms.png}
  \end{center}
  \caption{Histograma da variável de saída}
  \label{hist_Y}
  \end{figure}

Deve-se apontar a forma multimodal da variável de saída.
Nos podemos já ter em mente que uma regressão linear não vai dar certo.

%----------------------------------------------------------------------------------------

\subsection{Matriz de correlação}
A matriz de correlação é colocada na figura~\ref{correlation_matrix}

\begin{figure}[H] % image flottante, i.e latex selectione le meilleur placement
\begin{center}
\includegraphics[width=10cm,height=10cm]{/home/gjeusel/projects/intelComp/energy-efficiency/images/correlation_matrix_reg_prob with X6.png}
\end{center}
\caption{Matriz de correlação}
\label{correlation_matrix}
\end{figure}

As variáveis X1 (Relative Compactness) e X2 (Surface Area) são inversamente proporcional com um coeficiente de correlação igual a -1.
Olhando no papel dos autores, nos podemos encontrar a explicação desse resultado:
nos valores escolhidos para as simulações, eles fizeram a hipótese de um volume total dos edifícios constantes.
Isto acarreta num relação analítica que liga X1 com X2.
Observa-se o mesmo fenômeno com "X4 Roof Area" e "X5 Overall Height".

As variáveis de entradas “X4 Roof Area” e “X5 Overall Height” são variáveis altamente correlacionadas com a variável de saída.
Elas vão ter um efeito importante na predição do y.

No entanto, nos vemos que a variavel "X5 Orientation" que pode ser retirada devido ao fato que ela não é correlacionada com nenhuma otras variavel:
ela não da informações relevantes.
Nós removemos essa variável para a continuação do estudo.


%----------------------------------------------------------------------------------------
%    Atividade preditiva: regressões
%----------------------------------------------------------------------------------------
\newcommand{\norme}[1]{\left\Vert #1\right\Vert}
\newcommand{\abs}[1]{\left\vert #1\right\vert}
\newpage
\section{Atividade preditiva: Regressões}

%----------------------------------------------------------------------------------------
\subsection{Metodologia seguida}

Para cada modelo, será apresentado rapidamente o conceito matemático, e dado o gráfico (y medido, y predito) obtido.\newline
A discussão sobre o desempenho de cada modelo será feita no final da secção, comparando todas as métricas de validação obtidas.\newline

As métricas de validação usadas são:
\begin{itemize}
\item o \emph{coeficiente de determinação} R2:
    \begin{equation}
    R^{2} = \frac{\sum_{t=1}^{N} (\hat{y}(t) - \bar{y})^2}{\sum_{t=1}^{N} (y(t) - \bar{y})^2}
    \end{equation}
\item \emph{raiz quadrada do EMQ}, conhecida como RMS:
    \begin{equation}
    RMS = \sqrt{\frac{1}{N}\sum_{t=1}^{N} (y(t) - \hat{y}(t))^2}
    \end{equation}
\item \emph{erro absoluto médio percentual} MAPE:
    \begin{equation}
    MAPE = \frac{1}{N}\sum_{t=1}^{N} \abs{\frac{y(t) - \hat{y}(t)}{y(t)}}
    \end{equation}
\end{itemize}

$\hat{y}(t)$ é a previsão de y calculada pelo modelo de regressão no ponto $x(t)$.\newline

Finalmente, é importante de precisar que todos os $\hat{y}$ computados serão a união dos resultados de \emph{predições cruzadas} de \emph{10 ciclos}.
Isto quer significar que o conjunto de dados vai ser dividido em 10 subconjuntos.
Em cada ciclo (por um total de 10), o modelo é ajustado utilizando 9 subconjuntos e a saída é estimada por o subconjunto restante.
No fim, todas as estimativas serão concatenadas de maneira que nos temos uma estimativa da saída para cada registros.\newline
As estatísticas de validação serão calculadas com esse $\hat{y}$.

%----------------------------------------------------------------------------------------
\subsection{Modelo Linear}

No modelo linear, a estimativa $\hat{y}$ da variavel de saída é procurado usando a forma seguinte:
\begin{equation}
\hat{y}(t) = f(x(t),{\theta}) = \hat{x}(t){\theta}^{T} = \sum_{t=1}^{N} \hat{x}_i(t){\theta}_i
\end{equation}
com:
\begin{itemize}
\item $\hat{x}(t) = [1, h_1(x(t)), ..., h_N(x(t))]$ os regressores e $h_i(x(t))$ as funções de base
\item ${\theta} = ({\theta}_1, ...,{\theta}_N)$ o vetor de parâmetros
\end{itemize}

\medskip
Deve-se minimizar a função de custo, chamada de Erro Médio Quadrático para ajustar os parâmetros:
\begin{equation}
EMQ({\theta}) = \frac{1}{N}\sum_{t=1}^{N} (y(t) - \hat{y}(t))^{2}
\end{equation}

%----------------------------------------------------------------------------------------
\subsubsection{Modelo Linear de primeira ordem}
Nesse modelo, os regressores são as próprias variáveis de entrada: $\hat{x}(t) = [1, x(t)]$, i.e $h_i = Id$.

O gráfico dos valores preditivas é o seguinte:
\begin{figure}[H] % image flottante, i.e latex selectione le meilleur placement
\begin{center}
\includegraphics[width=8cm,height=8cm]{/home/gjeusel/projects/intelComp/energy-efficiency/images/graph_cv_predict_Linear.png}
\end{center}
\caption{Predicted vs measured - Linear first order}
\label{PvsM_Linear_First_Order}
\end{figure}

%----------------------------------------------------------------------------------------
\subsubsection{Modelo Linear Polinomial de grau r}
Nesse modelo, as funções de base são de forma polinômial: $h_i(x(t)) = x(t)^{i})$, com r o grau do polinômio.

O que dá: $\hat{x}(t) = [1, x(t), x(t)^{2}, ..., x(t)^{r}]$ como regressores.

Um estudo sobre a influência do grau escolhida do polinômio foi feita.
As estatísticas de validação obtidas para cada grau é dado pela tabela~\ref{tabela_inf_grau} e plotada na figura~\ref{graph_inf_grau}.\newline

\begin{figure}[H] % image flottante, i.e latex selectione le meilleur placement
\begin{center}
\includegraphics[width=\textwidth]{/home/gjeusel/projects/intelComp/energy-efficiency/images/Table of scores per Polynomial Regressor model.png}
\caption{Tabela influência grau - desempenho}
\label{tabela_inf_grau}
\bigskip
\bigskip
\includegraphics[width=\textwidth]{/home/gjeusel/projects/intelComp/energy-efficiency/images/Polynomial regressor scores vs deg.png}
\caption{Gráfico influência grau - desempenho}
\label{graph_inf_grau}
\end{center}
\end{figure}

Os scores R2 e RMS obtidos para os polinômios de grau 4 e 5 não faze nenhum sentido.
Eu não consegui entender onde ficou o problema na hora da computação deles.
Lendo a documentação scikit da função r2\_score, um valor negativo significa que o modelo é \emph{"arbitrarily worse"}.

No enquanto, nos podemos observar que os graus 2 e 3 são bem parecidos em termo de qualidade de modelagem.
Além disso, para graus superiores nos podemos assumir uma situação de overfitting, com uma complexidade da modelagem superior ao que é preciso.\newline
É interessante de notar que para o polinômio de grau 1, a solução do modelo linear simple é encontrada.

O gráfico dos valores preditivas para o Modelo Linear Polinomial de grau 3 é o seguinte:
\begin{figure}[H] % image flottante, i.e latex selectione le meilleur placement
\begin{center}
\includegraphics[width=8cm,height=8cm]{/home/gjeusel/projects/intelComp/energy-efficiency/images/graph_cv_predict_Polynomial deg 3.png}
\end{center}
\caption{Predicted vs measured - Linear Polinomial deg 3}
\label{PvsM_Linear_Polinomial_deg3}
\end{figure}


%----------------------------------------------------------------------------------------
\subsubsection{Modelo Linear de primeira ordem com regularização de Tikhonov}
Usando a regularização de Tikhonov, chamada "Ridge regression" em inglês, a função de custo que tem que ser minimizada é da forma:
\begin{equation}
EMQR({\mu},{\theta}) = \frac{1}{N}\sum_{t=1}^{N} (y(t) - \hat{y}(t))^{2} + {\mu}\norme{{\theta}}^{2}
\end{equation}
Ela é chamada de Erro Médio Quadrático Regularizado.
Isto é uma técnica de controle de complexidade do modelo através da aplicação de uma penalidade sobre o vetor de parametros.

A influência da penalidade escolhida (alpha) sobre as métricas de validação pode ser deduzido da figura~\ref{graph_inf_alpha}.\newline

\begin{figure}[H] % image flottante, i.e latex selectione le meilleur placement
\begin{center}
\includegraphics[width=\textwidth]{/home/gjeusel/projects/intelComp/energy-efficiency/images/Ridge regressor scores vs alpha.png}
\caption{Gráfico influência alpha - desempenho}
\label{graph_inf_alpha}
\end{center}
\end{figure}

Dá para ver que os resultados são melhorados com um alpha pequeno, mas que esse ganho de desempenho é quase insignificante.
Além disso, quando o alpha tende para 0, nos convergemos para a solução do problema linear de preimeira ordem sem regularização.
Isto quer dizer que para esse conjunto de dados, nos não podemos esperar obter melhores resultados usando essa regularização.

O gráfico dos valores preditivas para o Modelo Linear de primeira ordem com regularização de Tikhonov é o seguinte:
\begin{figure}[H] % image flottante, i.e latex selectione le meilleur placement
\begin{center}
\includegraphics[width=8cm,height=8cm]{/home/gjeusel/projects/intelComp/energy-efficiency/images/graph_cv_predict_Linear + SVD regularization alpha=0.001.png}
\end{center}
\caption{Predicted vs measured - Ridge alpha=0.001}
\label{PvsM_Linear_Ridge}
\end{figure}


%----------------------------------------------------------------------------------------
\subsection{Random Forest Regressor}

\subsubsection{Apresentação do Random Forest}
A ideá seguida nesse modelo é simplesmente de treinar o modelo com um número de árvore de decisão grande com características aleatórias,
e de pegar a media para melhorar a capacidade preditiva.
Por lembrete, a árvore de decisão é uma estrutura de dados definida recursivamente como:
\begin{itemize}
\item Um nó folha que contém o valor de uma classe
\item Um nó decisão que contém um teste sobre algum atributo.
\item Para cada resultado do teste existe uma aresta para uma subárvore, que tem a mesma estrutura da árvore.
\end{itemize}

Inicialmente designado para problemas de classificação, nos podemos utilizar ele assumindo que as nó folhas contem os valores da variável de saída,
o teste como uma verificação da distancia entre o valor previsto e o valor querido.
A figura~\ref{exemple_scikit_decisionTree} do website scikit ilustra o algoritmo de ávore de decisão usada para aproximar uma curva de seno com ruído.

\begin{figure}[H] % image flottante, i.e latex selectione le meilleur placement
\begin{center}
\includegraphics[width=8cm,height=8cm]{/home/gjeusel/projects/intelComp/energy-efficiency/images/decision tree algo scikit exemple.png}
\end{center}
\caption{Exemple : decision tree used to estimate a sine curve with additional noisy observation}
\label{exemple_scikit_decisionTree}
\end{figure}

\subsubsection{Estudo da influência do número de ávore escolhido}

Sem limite de profundidade, as métricas de validação obtidas em função do número de árvore escolhido é colocado na figura~\ref{graph_inf_numtree}.\newline

\begin{figure}[H] % image flottante, i.e latex selectione le meilleur placement
\begin{center}
\includegraphics[width=\textwidth]{/home/gjeusel/projects/intelComp/energy-efficiency/images/Random Forest regressor scores vs num trees.png}
\caption{Gráfico influência número de árvore - desempenho}
\label{graph_inf_numtree}
\end{center}
\end{figure}

Nos podemos observar um variabilidade importante sendo as métricas RMS e MAPE, reflectindo o comportamento aleatório do algoritmo.
No enquanto, a métrica R2 não muda muito em função do número de árvore escolhido.
Isto é possivelmente devido ao fato que na formula do calculo do coeficiente de determinação,
é pegado o valor média da saída $\bar{y}$, o que acarreta suavizar o comportamento aleatório.

\subsubsection{Gráfico dos valores preditivas}

O gráfico dos valores preditivas para o algoritmo de Random Forest com 10 árvores é o seguinte:
\begin{figure}[H] % image flottante, i.e latex selectione le meilleur placement
\begin{center}
\includegraphics[width=8cm,height=7cm]{/home/gjeusel/projects/intelComp/energy-efficiency/images/graph_cv_predict_Random Forest 10 trees.png}
\end{center}
\caption{Predicted vs measured - RandomForest n\_trees=10}
\label{PvsM_RandomForest}
\end{figure}


%----------------------------------------------------------------------------------------
\subsection{Conjunto de resultados e comparações}

As estatísticas de validação de todos os modelos anteriormente apresentados são colocado na tabela~\ref{table_all_results}.

\begin{figure}[H] % image flottante, i.e latex selectione le meilleur placement
\begin{center}
\includegraphics[width=\textwidth]{/home/gjeusel/projects/intelComp/energy-efficiency/images/Table of validation metrics per Regressor model without X6.png}
\end{center}
\caption{Métricas de validação em função do modelo}
\label{table_all_results}
\end{figure}

Nos podemos concluir que esse conjunto de dados não é bem modelado por modelos lineares simples.
No enquanto o modelo linear polinômial de grau 3 tem aproximativamente o mesmo desempenho que o modelo de RandomForest.
Seja bem de comparar esses modelos com modelos de redes neurais, mas eu não consegui instalar a última versão de scikit que tem esses algoritmos de rede neural.


%----------------------------------------------------------------------------------------
%    Estudos de Regressões complementares
%----------------------------------------------------------------------------------------
\newpage
\section{Estudos de Regressões complementares}

%----------------------------------------------------------------------------------------
\subsection{Influência da variável "X6 Orientation"}

A mesma tabela de resultados pegando em conta a variável X6:

\begin{figure}[H] % image flottante, i.e latex selectione le meilleur placement
\begin{center}
\includegraphics[width=\textwidth]{/home/gjeusel/projects/intelComp/energy-efficiency/images/Table of validation metrics per Regressor model.png}
\end{center}
\caption{Métricas de validação em função do modelo - com a variável X6}
\label{table_all_results_withX6}
\end{figure}

Comparando com a tabela~\ref{table_all_results}, nos podemos certificar que a variável X6 tem uma influencia desprezível.


%----------------------------------------------------------------------------------------
%    Tables:
%----------------------------------------------------------------------------------------
\newpage
\listoftables
\listoffigures

\newpage
\bibliography{biblio} % mon fichier de base de données s'appelle bibli.bib
\bibliographystyle{unsrt} % unsrt: par ordre d'apparition dans le text

\end{document}
