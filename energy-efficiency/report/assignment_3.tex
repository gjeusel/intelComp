%----------------------------------------------------------------------------------------
%    PACKAGES AND OTHER DOCUMENT CONFIGURATIONS
%----------------------------------------------------------------------------------------

\documentclass[paper=a4, fontsize=11pt]{article} % A4 paper and 11pt font size

% Brazilian encoding :
\usepackage[brazilian]{babel} % English language/hyphenation
\usepackage[T1]{fontenc} % Use 8-bit encoding that has 256 glyphs
\usepackage[utf8]{inputenc}

\usepackage{graphicx} % to insert png
\usepackage{fourier} % Use the Adobe Utopia font for the document - comment this line to return to the LaTeX default
\usepackage{amsmath,amsfonts,amsthm} % Math packages

\usepackage{lipsum} % Used for inserting dummy 'Lorem ipsum' text into the template

%\usepackage{geometry}
%\geometry{
% %a4paper,
% %total={170mm,257mm},
% %left=25mm,
% %top=25mm,
%}

\usepackage{adjustbox} % for better figure positionning
\usepackage[a4paper]{geometry} % for geometry changes in only one page (\newgeometry{...})

\usepackage{sectsty} % Allows customizing section commands
%\allsectionsfont{\centering \normalfont\scshape} % Make all sections centered, the default font and small caps

\usepackage{fancyhdr} % Custom headers and footers
\pagestyle{fancyplain} % Makes all pages in the document conform to the custom headers and footers
\fancyhead{} % No page header - if you want one, create it in the same way as the footers below
\fancyfoot[L]{} % Empty left footer
\fancyfoot[C]{} % Empty center footer
\fancyfoot[R]{\thepage} % Page numbering for right footer
\renewcommand{\headrulewidth}{0pt} % Remove header underlines
\renewcommand{\footrulewidth}{0pt} % Remove footer underlines
\setlength{\headheight}{13.6pt} % Customize the height of the header

\numberwithin{equation}{section} % Number equations within sections (i.e. 1.1, 1.2, 2.1, 2.2 instead of 1, 2, 3, 4)
\numberwithin{figure}{section} % Number figures within sections (i.e. 1.1, 1.2, 2.1, 2.2 instead of 1, 2, 3, 4)
\numberwithin{table}{section} % Number tables within sections (i.e. 1.1, 1.2, 2.1, 2.2 instead of 1, 2, 3, 4)

\usepackage{float}% If comment this, figure can moves to next page

\usepackage{indentfirst}% for indenting also the first paragraph after begin section
%\setlength\parindent{0pt} % Removes all indentation from paragraphs - comment this line for an assignment with lots of text

%----------------------------------------------------------------------------------------
%    TITLE SECTION
%----------------------------------------------------------------------------------------

\newcommand{\horrule}[1]{\rule{\linewidth}{#1}} % Create horizontal rule command with 1 argument of height

\title{
\normalfont \normalsize
\textsc{Universidade Federal do Rio de Janeiro} \\ [25pt] % Your university, school and/or department name(s)
\horrule{0.5pt} \\[0.4cm] % Thin top horizontal rule
\huge Aprendizado de Máquina\\Eficiência energética dos edifícios \\ % The assignment title
\horrule{2pt} \\[0.5cm] % Thick bottom horizontal rule
}

\author{Aluno: Guillaume Jeusel} % Your name

%\Professor: Alexandre G. Evsukoff

\date{\normalsize\today} % Today's date or a custom date

\date{\parbox{\linewidth}\bigskip%
  Professor: Alexandre G. Evsukoff \endgraf\bigskip
  Disciplina: Inteligência Computacional \endgraf\bigskip
  {\centering\normalsize\today\endgraf}
  }

\begin{document}

\maketitle % Print the title
\newpage

\tableofcontents % Print table of content
\newpage

%----------------------------------------------------------------------------------------
%    Introduction :
%----------------------------------------------------------------------------------------

\section{Introdução}

\subsection{Problema}
Com uma demanda de energia sempre crescente nosso mundo, o problema de economia de energia é colocado no centro das preocupações.
O conceito de \emph{négaWatt} \cite{ref_negawatt} se base sobre a ideá que é mais barato economizar energia do que comprar-lho.
E um campo cujo desperdício de energia continua a ser importante é o edifício.

Por conseguinte, as investigações na área do desempenho energético dos edifícios cresceu muito recentemente;
uma acção prioritária que as sociedades deve ter em mente é a redução do consumo de energia dos novos edifícios, também como a renovação dos antigos.
A propósito, a legislação sobre o desempenho energético dos edifícios é sempre mas exigente,
especificamente nos países europeus com a directiva 2002/91/CE limitando o consumo de energia dos edifícios \cite{ref_europ_energ}.

%------------------------------------------------

\subsection{Conhecimento Prévio}
Para o design desses edifícios, é necessário a computação dos termos chamados \emph{“Heat Load” e “Cooling Load”}
(que pode ser traduzido pelo “carga de aquecimento” e “carga de arrefecimento” respetivamente).
Eles são diretamente ligados à especificação dos equipamentos responsáveis para manter uma temperatura confortável, e então ao consumo energético.
Esses coeficientes são dependentes das características geométricas dos edifícios, como também do clima local e do uso deles (industrial, casal …).

Existem muitos diferentes software de simulação que são eficientes para prever o consumo energético dos edifícios em projeto com uma precisão aceitável.
Eles resolvam as equações diferencias da termodinmica aplicada a uma geometria particular.
No enquanto, essas simulações podem demorar muito tempo, sem mencionar que quando um parmetro é mudado, a simulação deve ser reinicializada desde ao início.

Desse fato, um interesso crescente sobre o uso das técnicas de aprendizado de máquinas nasci.
A ideá é a seguinte: suponho que você tem um banco de dados recente com as características e cargas de um grande número de edifícios,
o uso de estatísticas e aprendizado de máquinas pode reduzir o tempo de computação e facilitar o experimento de diversos parmetros.
Nos podemos pensar até criar um banco de dados com os diferentes resultados de simulação,
e depois prever o desempenho energético de um novo edifício com interpolação dos resultados que nos já temos.

Isto foi a ideá do engenheiro civil \emph{Angeliki Xifara} e do matemático \emph{Athanasios Tsanas} da universidade de Oxford.
Usando o software Ecotect, um conjunto de dado foi criado da simulação do desempenho energético para 768 geometrias de edifícios,
assumindo uma localização em Atena, Grécia e um uso residencial com sete pessoas.
Nos vamos estudar esse banco de dados.

Para ter mais informações sobre as hipóteses de simulação, deve-se referir ao papel deles \cite{ref_Athanasios}.


%----------------------------------------------------------------------------------------
%    Descrição dos dados :
%----------------------------------------------------------------------------------------

\section{Descrição dos dados}

\subsection{Dados}
O dataset é tirado do web-site UCI – Machine Learning Repository \cite{ref_UCI}.
A figura~\ref{dados_resumo} contem um resumo geral desse conjunto de dados.

\begin{figure}[H] % image flottante, i.e latex selectione le meilleur placement
\begin{center}
\includegraphics[width=\textwidth]{/home/gjeusel/projects/intelComp/energy-efficiency/resumo_dos_dados.png}
\end{center}
\caption{Características dos dados}
\label{dados_resumo}
\end{figure}

\medskip
Ele é composto de 768 registros e tem 8 variáveis de entrada e 2 de saídas que são as seguintes:

\begin{table}[H]
\caption{Mathematical representation of the input and output variables}
\medskip
\renewcommand\arraystretch{1.3}
\renewcommand\tabcolsep{0pt}
  \begin{tabular*}{1\linewidth}{@{\extracolsep{\fill}}ccc}
    \hline
    Mathematical representation & Input or output variable & Number of possible values \\
    \hline
    X1 & Relative Compactness & 12 \\
    X2 & Surface Area & 12 \\
    X3 & Wall Area & 7 \\
    X4 & Roof Area & 4 \\
    X5 & Overall Height & 2 \\
    X6 & Orientation & 4 \\
    X7 & Glazing Area & 4 \\
    X8 & Glazing Area Distribution & 6 \\
    y1 & Heating Load & 586 \\
    y2 & Cooling Load & 636 \\
  \end{tabular*}\par\medskip
\label{tab:priors}
\end{table}

É importante de notar que as variáveis de entradas são descontinuidades, e que não tem valores ausentes.

%------------------------------------------------

\subsection{Estatísticas do conjunto de dados}

\subsubsection{Variáveis de entradas}

  \begin{figure}[H] % image flottante, i.e latex selectione le meilleur placement
  \begin{center}
\adjustbox{trim={.0\width} {.0\height} {0.0\width} {0.069\height},clip}{
  \includegraphics[width=\textwidth]{/home/gjeusel/projects/intelComp/energy-efficiency/Xvar_desc_part1.png}
}
\adjustbox{trim={.0\width} {.0\height} {0.0\width} {0.069\height},clip}{
  \includegraphics[width=\textwidth]{/home/gjeusel/projects/intelComp/energy-efficiency/Xvar_desc_part2.png}
}
  \end{center}
  \caption{Estatísticas das variáveis de entradas}
  \label{statis_X}
  \end{figure}

Olhando para a valor media das variáveis na figura, nos podemos observar uma diferencia de escala entre as variaveis.
Na verdade, a maioria dessas variáveis não têm a mesma unidade:
\begin{itemize}
\item 0 < X1 < 1 sem unidades
\item X2, X3, X4 em metros quadrados
\item X5 em metros
\item X7 em percentagem
\item X8 em metros quadrados
\end{itemize}

Quando nos vamos comparar essas variáveis entre elas, nos vamos ter que padronizar elas.
O escolho da métrica de Z-score foi feito, embora que a padronização min-max pudesse ser feita,
devido ao fato que a amostra não tem outliers como nos vamos ver na secção seguinte.

Formula de padronização Z-score applicada:
\begin{equation}
\hat{X}_{i}(t) = \frac{X_{i}(t) - \bar{X}_{i}}{\hat{\sigma}_{i}}
\end{equation}

\subsubsection{Variáveis de saídas}

  \begin{figure}[H] % image flottante, i.e latex selectione le meilleur placement
  \begin{center}
\adjustbox{trim={.0\width} {.0\height} {0.0\width} {0.069\height},clip}{
  \includegraphics[width=0.6\textwidth]{/home/gjeusel/projects/intelComp/energy-efficiency/Yvar_desc.png}
}
  \end{center}
  \caption{Estatísticas das variáveis de saídas}
  \label{statis_Y}
  \end{figure}

As variáveis de saída aparecem bastante semelhantes. Nos vamos verificar isso com os histogramas.

%----------------------------------------------------------------------------------------

\subsection{Detecção de outliers}

\subsubsection{Box-plot}

  \begin{figure}[H] % image flottante, i.e latex selectione le meilleur placement
  \begin{center}
  \adjustbox{trim={.0\width} {.0\height} {0.0\width} {0.069\height},clip}{
    \includegraphics[width=\textwidth]{/home/gjeusel/projects/intelComp/energy-efficiency/Xvar_Znormalized_boxplot.png}
  }
  \end{center}
  \caption{Box-plot das variáveis de entradas Z-padronizadas}
  \label{boxplot_X}
  \end{figure}

  \begin{figure}[H] % image flottante, i.e latex selectione le meilleur placement
  \begin{center}
  \adjustbox{trim={.0\width} {.0\height} {0.0\width} {0.069\height},clip}{
    \includegraphics[width=\textwidth]{/home/gjeusel/projects/intelComp/energy-efficiency/Yvar_Znormalized_boxplot.png}
  }
  \end{center}
  \caption{Box-plot das variáveis de saídas Z-padronizadas}
  \label{boxplot_Y}
  \end{figure}

Pela leitura desses gráficos, nenhuma das variáveis aparecem conter valores aberantes.
Mas deve ser considerado todas as variáveis simultaneamente com os métodos baseados em distancia para ser capaz de comprovar esse resultado.
Nos vamos nos concentrar apenas nas variáveis de entrada na proxima seção.

\subsubsection{A métrica euclideana}

A métrica de distancia \emph{Euclidiana} (ou norma $L_2$) é a extensão da fórmula clássica da geometria para \emph{p} dimensões:
\begin{equation}
dist_{E}(v,u) = ||v-u|| = \sum_{i=1}^{p} (v_i - u_i)
\end{equation}

A matriz de distancias obtida é colocada na figura~\ref{dist_eucl_matrix}, e o gráfico das médias de distancias na figura~\ref{dist_eucl_graph}.

  \begin{figure}[H] % image flottante, i.e latex selectione le meilleur placement
  \begin{center}
  \includegraphics[width=\textwidth]{/home/gjeusel/projects/intelComp/energy-efficiency/distance_matrix_euclidean_Z_normed.png}
  \end{center}
  \caption{Matriz de distancias euclideana com variaveis Z-padronizadas}
  \label{dist_eucl_matrix}
  \end{figure}

  \begin{figure}[H] % image flottante, i.e latex selectione le meilleur placement
  \begin{center}
  \includegraphics[width=\textwidth]{/home/gjeusel/projects/intelComp/energy-efficiency/graph_mean_euclidean_dist_Znormed.png}
  \end{center}
  \caption{Médias de distancias Euclidiana em ordem crescente}
  \label{dist_eucl_graph}
  \end{figure}


\subsubsection{A métrica Mahalanobis}

Distancia de \emph{Mahalanobis} é a distancia geométrica ponderada pelo inverso da matriz de covariancias estimada no conjunto de dados:
\begin{equation}
dist_{\Sigma}(v,u) = \sqrt{(v-u)\hat{\Sigma}^{-1}(v-u)^T}
\end{equation}

A matriz de distancias obtida é colocada na figura~\ref{dist_maha_matrix}, e o gráfico das médias de distancias na figura~\ref{dist_maha_graph}.

  \begin{figure}[H] % image flottante, i.e latex selectione le meilleur placement
  \begin{center}
  \includegraphics[width=\textwidth]{/home/gjeusel/projects/intelComp/energy-efficiency/distance_matrix_mahalanobis.png}
  \end{center}
  \caption{Matriz de distancias Mahalanobis}
  \label{dist_maha_matrix}
  \end{figure}

  \begin{figure}[H] % image flottante, i.e latex selectione le meilleur placement
  \begin{center}
  \includegraphics[width=\textwidth]{/home/gjeusel/projects/intelComp/energy-efficiency/graph_mean_mahalanobis_dist.png}
  \end{center}
  \caption{Médias de distancias Mahalanobis em ordem crescente}
  \label{dist_maha_graph}
  \end{figure}

\subsubsection{Discussão dos resultados}

Como um lembrete, o gráfico das médias ordenadas é obtida pela fórmula:
\begin{equation}
\overline{dist}_i = \frac{1}{N}*\sum_{t=1}^{N} (distance(u_i, v_t))
\end{equation}

Podemos considerar que não há nenhum registro aberrantes, porque nenhum é fortemente afastado dos outros (seja com a medida euclideana ou a medida Mahalanobis).
Além disso, não é surpreendente porque as variáveis de entradas foram selecionadas pelo engenheiro que fiz as simulações, então ele escolheu uma faixa responsável para cada um delas.

É importante de notar que a distancia de Mahalanobis é adequada para medir a separação entre um conjunto de dados com
variáveis geradas pela distribuição normal multivariada.
No enquanto, nesse conjunto de dado, as distribuições não parecem a ela como nos vamos ver nos histogramas seguintes.

%----------------------------------------------------------------------------------------

\subsection{Distribuições}

\subsubsection{Histogramas das variáveis não padronizadas}

  \begin{figure}[H] % image flottante, i.e latex selectione le meilleur placement
  \begin{center}
  \includegraphics[width=\textwidth]{/home/gjeusel/projects/intelComp/energy-efficiency/Xvar_histograms_part1.png}
  \includegraphics[width=\textwidth]{/home/gjeusel/projects/intelComp/energy-efficiency/Xvar_histograms_part2.png}
  \end{center}
  \caption{Histogramas das variáveis de entradas}
  \label{hist_X}
  \end{figure}

  \begin{figure}[H] % image flottante, i.e latex selectione le meilleur placement
  \begin{center}
  \includegraphics[width=0.8\textwidth]{/home/gjeusel/projects/intelComp/energy-efficiency/Yvar_histograms.png}
  \end{center}
  \caption{Histogramas das variáveis de saídas}
  \label{hist_Y}
  \end{figure}

Podemos então confirmar que as variáveis de saídas são parecidas.
No estudo futuro de regressão, será interessante de somente considerar um delas no primeiro lugar.

Uma outra coisa que deve ser apontada é a forma multimodal das variáveis de saída.
Nos podemos já ter em mente que uma regressão linear não vai dar certo.

\subsubsection{Gráficos de Projeção}
O gráficos de projeção é colocada na figura~\ref{scatter_matrix}

\newgeometry{top=0.5cm}
  \begin{figure}[p] % image flottante, i.e latex selectione le meilleur placement
  \begin{center}
  \includegraphics[width=1.7\textwidth, angle=-90]{/home/gjeusel/projects/intelComp/energy-efficiency/Scatter_matrix.png}
  \end{center}
  \caption{Gráficos de Projeção}
  \label{scatter_matrix}
  \end{figure}
\restoregeometry

Parece que as variáveis X1 e X2 são bem correlacionadas, e que a variável X5 não dá muito mais informações de que as outras.
Isto é comprovado com a matriz de correlação.

%----------------------------------------------------------------------------------------

\subsection{Matriz de correlação}
A matriz de correlação é colocada na figura~\ref{correlation_matrix}

  \begin{figure}[H] % image flottante, i.e latex selectione le meilleur placement
  \begin{center}
  \includegraphics[width=\textwidth]{/home/gjeusel/projects/intelComp/energy-efficiency/correlation_matrix.png}
  \end{center}
  \caption{Matriz de correlação}
  \label{correlation_matrix}
  \end{figure}

Efetivamente, as variáveis X1 (Relative Compactness) e X2 (Surface Area) são inversamente proporcional com um coeficiente de correlação igual a -1.
Olhando no papel dos autores, nos podemos encontrar a explicação desse resultado:
nos valores escolhidos para as simulações, eles fizeram a hipótese de um volume total dos edifícios constantes.
Isto acarreta num relação analítica que liga X1 com X2.

Nos podemos observar também que X4 (Roof Area) é bem correlacionado com X5 (Overall Height), provavelmente devido da mesma hipótese.

Finalmente, as duas variáveis de saída y1 e y2 são fortemente correlacionadas.

%----------------------------------------------------------------------------------------
%    Discussão :
%----------------------------------------------------------------------------------------

\section{Discussão}

Nos fizemos a analisa de um conjunto de dados com pouco variáveis, que já estava bem condicionada (sem valores ausentes nem outliers).
No enquanto, nos vimos que as distribuições não são trívias.

A próxima etapa do projeto vai ser de encontrar um modelo capaz de prever as variáveis y1 (Heat Load) e y2 (Cooling Load) em função das variáveis geométricas dos edifícios.
E nos já apontamos que uma regressão linear não parece ser um bom escolho de modelo por causa dessas distribuições foram do comum.

Nos podemos também comentar o fato de que o website UCI colocou esse dataset na categoria de exercícios de classificação.
A figura~\ref{energy_class_buildings} mostra a classificação oficial na Europa \cite{ref_class_energ}.
O dataset deve ser simplesmente considerado do ponto de vista de classificação energética dos edifícios.

\begin{figure}[H] % image flottante, i.e latex selectione le meilleur placement
\begin{center}
\includegraphics[width=\textwidth]{/home/gjeusel/Downloads/classification-energie-building.jpg}
\end{center}
\caption{Classificação energética dos edifícios na Europa}
\label{energy_class_buildings}
\end{figure}

%----------------------------------------------------------------------------------------
%    Tables :
%----------------------------------------------------------------------------------------
\newpage
\listoftables
\listoffigures

\newpage
\bibliography{biblio} % mon fichier de base de données s'appelle bibli.bib
\bibliographystyle{unsrt} % unsrt: par ordre d'apparition dans le text

\end{document}
